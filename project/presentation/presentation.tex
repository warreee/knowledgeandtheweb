\documentclass[nonav,sleutel]{beamer}
\usepackage[utf8]{inputenc}
\usepackage[T1]{fontenc}
\title{Politicians and Nobel Prizes}
\date[ISPN '80]{Prof. Dr. Bettina Berendt\\ Knowledge \& the Web 2015-2016}
\author{Katrien Laenen \and Gust Verbruggen \and Ward Schodts}

\usepackage{booktabs}
\usetheme{kuleuvenstijl}

\makeatletter
\def\beamer@tocaction@only#1{\only<.(1)>{\usebeamertemplate**{#1}}}
\define@key{beamertoc}{subsectionsonly}[]{\beamer@toc@subsectionstyle{show/only}\beamer@toc@subsubsectionstyle{show/shaded/hide}}

\newif\ifbeamer@pausebeforesubsections
\define@key{beamertoc}{pausebeforesubsections}[true]{\beamer@pausebeforesubsectionstrue}

\patchcmd{\beamer@tableofcontents}{\beamer@pausesectionsfalse}%
  {\beamer@pausesectionsfalse\beamer@pausebeforesubsectionsfalse}{}{}

\patchcmd{\beamer@subsectionintoc}{\ifbeamer@pausesubsections\pause\fi}%
  {\ifbeamer@pausesubsections\pause\else%
   \ifbeamer@pausebeforesubsections\ifnumequal{#2}{1}{\pause}{}\fi\fi}{}{}
\makeatother


\begin{document}


%%%%%%%%%%%%%%%%%%%%%%%%%%%%%%%%%%%%%%%%%%%%%%%%%%%%%%%%%%%%%%%%%%%%%%%%%%%%%%%%%%%%%%%%%%%%%%%

\begin{frame}
\titlepage
\end{frame}

%%%%%%%%%%%%%%%%%%%%%%%%%%%%%%%%%%%%%%%%%%%%%%%%%%%%%%%%%%%%%%%%%%%%%%%%%%%%%%%%%%%%%%%%%%%%%%%

\begin{frame}[noframenumbering]
\frametitle{Outline} 
  \tableofcontents[hideallsubsections
  ]

\end{frame}

\AtBeginSection[]
{
 \begin{frame}<beamer>
 \frametitle{Outline}
 \tableofcontents[subsectionsonly, pausesections]
 \end{frame}
}


%%%%%%%%%%%%%%%%%%%%%%%%%%%%%%%%%%%%%%%%%%%%%%%%%%%%%%%%%%%%%%%%%%%%%%%%%%%%%%%%%%%%%%%%%%%%%%%
\section{Introduction}
\begin{frame}
\frametitle{Introduction} 
\framesubtitle{Some history} 
\begin{theorem}
There is no largest prime number. \end{theorem} 
\begin{enumerate} 
\item<1-| alert@1> Suppose $p$ were the largest prime number. 
\item<2-> Let $q$ be the product of the first $p$ numbers. 
\item<3-> Then $q+1$ is not divisible by any of them. 
\item<1-> But $q + 1$ is greater than $1$, thus divisible by some prime
number not in the first $p$ numbers.
\end{enumerate}
\end{frame}

%%%%%%%%%%%%%%%%%%%%%%%%%%%%%%%%%%%%%%%%%%%%%%%%%%%%%%%%%%%%%%%%%%%%%%%%%%%%%%%%%%%%%%%%%%%%%%%
\subsection{The research question}
\begin{frame}{A longer title}
\begin{itemize}
\item one
\item two
\end{itemize}

One can prove that
\[
	1 = 1
\]
\end{frame}

\begin{frame}{Blocks}
\begin{block}{Block title}
Block body.
\end{block}
\begin{example}
For clarity:
\begin{itemize}
	\item[$\rightarrow$] first bullet point \ldots
	\item[$\rightarrow$] second bullet  point \ldots
\end{itemize}
\end{example}
\end{frame}

\end{document}