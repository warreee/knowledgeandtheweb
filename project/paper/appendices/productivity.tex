\section{Discussion on work productivity}
\label{sec:productivity}

Whether we can use different proxy measures for a certain feature is a sensitive issue. For this reason, we here provide a more thorough discussion of the matter. First, we discuss why we have to use different categories of people for building the model and training it. This introduces the problem of having to proxy a feature using  different measures. Next, we discuss why we believe the chosen measures might be a good proxy for the productivity feature. Finally, we give a weakened version of the research question that is now more appropriately answered.

\subsection{Different datasets}

The assignment was to find a research question revolving around the knowledge base, enriched with information from other sources. Because we found an interesting knowledge base on Nobel Prizes, the posed research question wasn't a far fetch. The problem is that in order to build a model on who would win a Nobel Prize, we need positive and negative examples. The positive examples are the Nobel laureates, which come from a pool of scientists. All other scientists naturally serve as negative examples. Since a politician isn't eligible to receive such prize, we can't use them as negative examples. It is thus inevitable to use different examples in the training set and the testing set.

\subsection{Work productivity}

Apart from university rankings and popularity measures, work productivity seems like a reasonable feature to use in order to assess eligibility to receive an award. We reason that hard work will more likely result in making some scientific breakthrough, which increases the chances of someone winning a Nobel Prize. The productivity feature is not trivial to define, however. Furthermore, even if we can define it, all information needed to compute it has to be available for as many training and testing instances as possible. We therefore view all examples as \emph{persons} rather than \emph{scientists} or \emph{politicians}, having certain properties. How we define the work productivity property is the only difference between both datasets. For scientists, the number of publications seems like a decent way of measuring how productive they've been. For politicians, the most resemblant feature would have been the number of legislations they managed to get accepted. This information isn't publicly available, however, so we looked at something else. Coming straight from the ToE knowledge base, the number of speeches at european parliament seemed a good proxy measure. Both are numeric properties which we can normalise to $[0, 100]$.

\subsubsection{Work productivity duality}
An interesting duality exists in both measures. Having a lot of publications over a mediocre amount can be recognised by the model as either a \emph{good} or \emph{bad} property. It can for example mean that someone is just trying to publish a lot of less significant results, while a mediocre amount of significant results might be more important. For speeches the same happens, where a lot of speeches might just as well mean that a politician is only talking, without actually \emph{doing} much. This re\"inforces our belief that both might be a good proxy for the same feature.

\subsection{Weakened research question}
A more appropriate version of the research question is then
\begin{center}
	\emph{\textbf{\textsc{IF}} the number of publications and the number of speeches in European Parliament are reliable proxies for the work productivity of scientists and politicians respectively \textbf{\textsc{THEN}} which European politicians have a high chance of receiving a Nobel Prize?}
\end{center}


