\section{Linked open data}
\label{sec:lod}
\subsection{Sparql endpoints}
\begin{table}[H]
\centering
\begin{tabular}{l|l}
	\textbf{\textsc{Source}} & \textbf{\textsc{Sparql endpoint}} \\ \hline
	\rule{0pt}{4mm}DBPedia & \texttt{http://dbpedia.org/sparql/} \\
	ToE &  \texttt{http://linkedpolitics.ops.few.vu.nl} \\
	& \texttt{/sparql/} \\
	Nobel Prize & \texttt{http://data.nobelprize.org/sparql}
\end{tabular}
\caption{Overview of used Sparql endpoints.}
\end{table}
\subsection{Difficulites with LoD}
In this section we discuss our difficulties that we experienced with linked open data (LoD). These are worth mentioning because they influenced our results.

\subsubsection{The language barrier}
The great thing about LoD is that it comes in different languages, what means that you have a bigger and a broader information source.
Unfortunately it also brings a lot of difficulties with it.
Not only do the names of people, institutions and other features differ from each other, also the arguments that you give in the queries are sometimes spelled differently.
All these different namings made it more difficult to match the the right names with the right features.
That's why we decided to filter everything out that wasn't available in English. One could say that this would create a bias. We decided not to worry about this because the biggest part of LoD is available in English.

\subsubsection{It just doesn't work sometimes}

