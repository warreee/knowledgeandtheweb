\documentclass[11pt,journal,compsoc]{IEEEtran}

\makeatletter
\renewcommand{\@IEEEsectpunct}{\ \,}% Modified from {:\ \,}
\makeatother

% *** GRAPHICS RELATED PACKAGES ***


% *** MATH PACKAGES ***
\usepackage[cmex10]{amsmath}
% Note that the amsmath package sets \interdisplaylinepenalty to 10000
% thus preventing page breaks from occurring within multiline equations. Use:
%\interdisplaylinepenalty=2500
% after loading amsmath to restore such page breaks as IEEEtran.cls normally
% does.



% *** SUBFIGURE PACKAGES ***
%\ifCLASSOPTIONcompsoc
%  \usepackage[caption=false,font=footnotesize,labelfont=sf,textfont=sf]{subfig}
%\else
%  \usepackage[caption=false,font=footnotesize]{subfig}
%\fi

% *** FLOAT PACKAGES ***
%
%\usepackage{fixltx2e}
\usepackage{float}
\usepackage{stfloats}
\usepackage{caption}
\captionsetup{
	font=scriptsize,
	labelfont=scriptsize,
}
\captionsetup[lstlisting]{justification=centering}


% *** PDF, URL AND HYPERLINK PACKAGES ***
%
\usepackage[colorlinks]{hyperref}
\hypersetup{
 urlcolor = blue
}

% Correct bad hyphenation here
\hyphenation{}

%----------------------------------------------------------------------------------------
%	CONFIGURATION OF CODE LISTINGS
%----------------------------------------------------------------------------------------
\usepackage[font=scriptsize,labelfont=scriptsize]{subcaption}
\usepackage{listings}
\usepackage{color}
 
\definecolor{codegreen}{rgb}{0,0.6,0}
\definecolor{codegray}{rgb}{0.5,0.5,0.5}
\definecolor{codepurple}{rgb}{0.58,0,0.82}
\definecolor{backcolour}{rgb}{0.95,0.95,0.92}
 
\lstdefinestyle{mystyle}{
    backgroundcolor=\color{backcolour},   
    commentstyle=\color{codegreen},
    keywordstyle=\color{magenta},
    numberstyle=\tiny\color{codegray},
    stringstyle=\color{codepurple},
    basicstyle=\footnotesize,
    breakatwhitespace=false,         
    breaklines=true,                 
    captionpos=b,                    
    keepspaces=true,                 
    numbers=left,                    
    numbersep=5pt,                  
    showspaces=false,                
    showstringspaces=false,
    showtabs=false,                  
    tabsize=2,
}
 
\lstset{style=mystyle}

%%%%%%%%%%%%%%%%%%%%%%%%%%%%%%%%%%%%%%%%%%%%%%%%%%%%%%%%%%%%%%%%%%%
%%%%%%%%% OWN:
%%%%%%%%%%%%%%%%%%%%%%%%%%%%%%%%%%%%%%%%%%%%%%%%%%%%%%%%%%%%%%%%%%%
\usepackage[
    backend=biber,
    url=true
]{biblatex}
\addbibresource{bibliography.bib}
\usepackage{hyperref}	
\usepackage{todonotes}	

\usepackage{multirow}

\usepackage[english]{babel}

\begin{document}
%

\title{Politicians and Nobel Prizes}

\author{Katrien Laenen,
        Ward Schodts
        and~Gust Verbruggen\\ 
\IEEEcompsocitemizethanks{\IEEEcompsocthanksitem katrien.laenen@student.kuleuven.be
\IEEEcompsocthanksitem ward.schodts@student.kuleuven.be
\IEEEcompsocthanksitem gust.verbruggen@student.kuleuven.be}%

}

% The paper headers
\markboth{Knowledge \& the Web, 2015-2016}%
{Shell \MakeLowercase{\textit{et al.}}: Bare Demo of IEEEtran.cls for Computer Society Journals}


% use for special paper notices
%\IEEEspecialpapernotice{(Invited Paper)}


\IEEEtitleabstractindextext{%
\begin{abstract}
In this paper we try to find the politicians which have a high change of receiving a Nobel Prize. 
\end{abstract}

\begin{IEEEkeywords}
Open data, linked data, semantic web, Nobel Prize, politician, European Parliament.
\end{IEEEkeywords}}


% make the title area
\maketitle

\IEEEdisplaynontitleabstractindextext

\IEEEpeerreviewmaketitle

\IEEEPARstart{S}{ince} 1901, every year there are 5 Nobel Prizes awarded to people all over the world. Each prize is for a specific field namely: Physics, Chemistry, Physiology or Medicine, Literature and Peace. In 1968 the Nobel Prize in Economic Sciences was added to this list and first awarded in 1969.
These prizes are rewarded in memory of Alfred Nobel \cite{nobelNobel}.
He was a chemist, engineer and inventor who during his lifetime made a fortune because of his large amount of inventions. His last will stated that his fortune be used to create a series of prizes for those who confer the "greatest benefit on mankind" \cite{wikiNobel}. 
Some of these prize have been received by politicians. Notable ones are: Nelson Mandela (Nobel Peace Prize 1993), Barack Obama (Nobel Peace Prize 2009) and Winston Churchill (Nobel Prize in Literature 1953) \cite{nobelList}.

We are interested in finding a model that is able to predict the probability of a politician to win a Nobel Prize.


\hfill \today

\subsection{Research question}
\subsubsection{Motivation}
There is a whole process of nominations and selection before the Nobel Prizes are handed out. This is a complex process that takes some time \cite{nobelNomination}.
Because of the non-triviality of who will win the prize, it would be interesting to have a model that is able to predict how likely a person is to win the prize based on his characteristics. We are mainly interested how this would apply to politicians. 
\subsubsection{Research question}
\par For this reason, the question which this paper attempts to answer is:
\begin{center}
\textit{"Which European politicians have a high chance of receiving a Nobel Prize?"}	
\end{center}
In this context, European politicians are those who settle in the European Parliament. With \textit{a high chance} we mean the probability that someone would get awarded a Nobel Prize during his lifetime.

\subsection{Structure of this paper}
After this introduction we will elaborate on our methodology. The first section explains how we tackled the problem. The following sections will elaborate on each of these steps. In Section~\ref{sec:data} we discuss the data needed to answer our question and where we get it. 
Section~\ref{sec:analysis} focuses on data analysis. This covers some statistics to assess the quality of our data and introduces the learned model. We then show the results of applying the model in Section~\ref{sec:results}. To conclude this all we make a conclusion in section 5. Finally, we suggest some things that can be interesting for future research.

\section{Methodology}
\label{sec:methodology}
To answer the research question we will use the approach described below. The next sections will explain the steps more thoroughly.
\begin{enumerate}
\item We think about what properties of previous Nobel Prize winners might be interesting to incorporate in learning a model. This is the feature selection step. Furthermore, we determine whether and where this information can be found.
\item We decide what the training dataset and research dataset should look like.
\item The available data is then collected by the aid of SPARQL queries and scraping webpages. This is combined into a training dataset and a research dataset. The training dataset can be used to learn a predictive model. The research dataset is used to answer the research question.
\item Some statistics are applied to the training dataset to assess its quality. For example, because collecting the data is prone to errors, some outlier detection and removal has to be performed.
\item The final training dataset is used to learn a model that is able to predict how likely a person is to win a Nobel Prize. A logistic regression model is used. Because it predicts the probability of particular outcomes, it is perfectly suited to answer the research question. The model's accuracy is assessed using cross-validation on the training dataset.
\item The final step is applying the learned model on the research dataset and discussing the results. 
\end{enumerate}


\section{Data}
\label{sec:data}

First, we determine features of people that might influence their probability of winning a Nobel Prize. They are elaborated in Section~\ref{ssec:features}. Next, we research whether and where this information can be found. The main data source are knowledge bases, also known as Linked Open Data in the Semantic Web. We can use SPARQL queries on an endpoint to easily retrieve structured data \cite{wc3SPARQL}. The used knowledge bases are discussed in Section~\ref{ssec:knowledgebases}. Needed information that is not (yet) linked has to be scraped from websites. These are discussed in Section~\ref{ssec:additional}. We use Perl scripts and regular expressions to extract the data from plain HTML text. All the retrieved data and all the used scripts can be found at the webpage for this paper: \url{kaw.wardschodts.ws}.

\subsection{Feature selection}
\label{ssec:features}
In order to determine whether someone will win a Nobel Prize, a set of reasonable features that the model can base its decisions on is needed. A summary of the chosen features is given here. 
\begin{itemize}
\item \textbf{Year of birth.}
Currently the average age of a laureate is 59 \cite{age}. Evidently, this plays a role in the decision if somebody makes a chance for a Nobel Prize. For example, young scientists are far less likely to win an award because they have yet to prove themselves.
\item \textbf{Alma mater.} We suppose that there's a link between the university where one graduated and his chances of winning a Nobel Prize. For example, admission to prestige universities is only granted to the best candidates. In order to extract a usable feature from an alma mater, we use its \emph{award score}, which is based on how many Nobel Prize and Field medal winners it produced. In case someone has studied at multiple universities that have a score higher than 0, the average is used. Unranked universities automatically have a score of 0 (which is an inherent property of how it is calculated), if this is the case for someone with multiple universities, then these universities are omitted from the average.
\item \textbf{Work productivity.} A measure of work productivity can also be useful to include. This is however not trivial to determine, as not many properties are available for the majority of scientists and politicians. In order to solve this, we attempted defining a \textbf{proxy} measure for work productivity for the training data and testing individuals. For the scientists and economists from the training dataset, their number of publications might be an appropriate measure for productivity. For politicians, the number of speeches they make are perhaps a good indication of how productive they are. \emph{Because this is a sensitive subject, Appendix \ref{sec:productivity} provides a thorough discussion of why we chose these features and how the research question is slightly weakened by our assumption. Furthermore, in Section~\ref{ssec:quality} on quality assessment, we show why we deem both to be a reasonable proxy by comparing their distributions.}
\item \textbf{Popularity.} A popularity measure is easier to define. Facebook automatically generates pages for individuals that exist on Wikipedia. Therefore all people that we consider have at least such automatically generated page. The number of likes these pages have can then serve as a measure for popularity. 
\end{itemize}

\subsection{Knowledge bases}
\label{ssec:knowledgebases}

Using SPARQL queries, knowledge bases are very easy to retrieve structured information from. Three knowledge bases that we used are listed and discussed below.

\begin{itemize}
\item{\textbf{DBpedia}} is a Linked Open Data source which contains information for just about everything. Coming straight from Wikipedia, it is updated daily and should thus be up-to-date all the time. Although it can be edited by anyone, data correctness is verified thoroughly by a team of content moderators. For this reason we get a large part of our training dataset from DBpedia. More specifically, we retrieve the year of birth, as well as alma mater and of course their full name. The name mostly serves as an identifier for linking data from multiple sources. Persons for which not all data is found are immediately omitted.

\item{\textbf{Nobelprize.org}} is the most complete Linked Open Data source concerning Nobel Prize winners. From 1901 on, it contains all Nobel Prizes that have been won. Moreover, it is maintained by the official organisation responsible for the Nobel Prize and should thus be complete and correct. Since it makes use of the FOAF and DBpedia ontologies, querying it is very easy. The most important feature that we retrieve from this source is of course whether someone has won a Nobel Prize. One thing missing is a \emph{sameAs:} link to DBPedia, which makes our task of linking it slightly harder. Our approach for merging this data with the other features is discussed in subsection \ref{ssec:merging}.

\item{\textbf{Talk of Europe}} is a Linked Open Data source about the European Parliament. The dataset covers all plenary debates held in the European Parliament between July 1999 and January 2014 and biographical information about the members of parliament. The politicians in this data source are the ones for which we would like to answer the research question. Using a \emph{sameAs:} relation to DBPedia, we can easily query for further information without needing exact name matches.
\end{itemize}

\noindent All SPARQL queries used to gather the needed information are also available at the webpage of this paper.
In appendix \ref{sec:lod} we provide the links to the specific endpoints and also some of our experiences of using Linked Open Data.

\subsection{Additional data}
\label{ssec:additional}

For the work productivity, university rankings and popularity features no linked data exists. We therefore scraped all needed information from webpages and linked them ourselves. This was done using Perl scripts and regular expressions, with varying difficulty for different sources.

\begin{itemize}
\item{\textbf{Shanghai Rankings. }} 	
The webpage at \cite{rankings} gives an overview of the Shanghai Rankings for 2015. 
It also includes  information on the criteria used to build this ranking, which includes the awards ranking. Universities that are not listed have an award ranking of 0. Because of the tabular form, retrieving the data was fairly straightforward.

\item{\textbf{Facebook. }} As mentioned, every person on Wikipedia has an automatically generated Facebook page. Querying Facebook can be done using the Graph API with simple HTTP GET requests \cite{graphAPI}. Specialised queries can be constructed using combinations of GET parameters, for which the results are returned in JSON format. Because of this, out of all non-linked sources, information from Facebook was by far the easiest to gather. One difficulty that is worth mentioning, is the limit of consecutive queries that can be send to the Facebook Graph API.
We noticed that after sending more or less 630 queries, the user token that is needed to retrieve the information expires. If you generate a new token you aren't allowed to immediatly start querying again. You have to wait for at least an hour. 
This is in fact an odd number of queries, if you read the documentation of the Graph API, only 200 queries per hour are allowed \cite{graphAPIlimit}.
For people with multiple pages, the number of likes for the most liked page is used. Because popular names might have pages not related to the scientist we are interested in, some results might be noisy. Such outliers have to be pruned afterwards.

\item{\textbf{Google Scholar. }} Google Scholar provides advanced search options, the most relevant for us is searching by author. In contrary to Facebook this not achieved by combining GET parameters, but by manually constructing the advanced search terms. For example, to search for publications by Albert Einstein the query  \emph{author:"Albert Einstein"} can be used. The resulting page shows the number of results, which we can then parse. Based on manual testing, Google's advanced search performs pretty well for finding only results for a given author. However, as before, people with the same name can give noisy results, so some outlier detection has to be performed. A second difficulty arose when Google appeared to block automated requests. Appendix \ref{sec:googlescholar} gives an overview of some methods we used in attempt to moderately successfully circumvent those restrictions.
\end{itemize}

\subsection{Overview}

The following table gives an overview of which feature information was collected from what sources. Linked Open Data sources are shown in \textbf{bold}.

\begin{table}[H]
\centering
\begin{tabular}{l|l}
	\textbf{\textsc{Feature}} & \textbf{\textsc{Source}} \\ \hline
	\rule{0pt}{4mm}Place of birth & \textbf{DBPedia} \\
	Year of birth & \textbf{DBPedia} \\
	University ranking & \textbf{DBPedia} + Shanghai Ranking 2015\\
	Work Productivity & Google Scholar, \textbf{ToE}\\
	Popularity & Facebook\\
	Nobel Prize& \textbf{Nobelprize.org}
\end{tabular}
\caption{Overview of used features and their respective sources.}
\end{table}

\subsection{Processing and merging}
\label{ssec:merging}
After collecting the data from different sources, it has to be combined in some way. For the most part, this was pretty easy by matching names. The hardest part was matching the universities of the people with those in the Shanghai rankings to retrieve the correct ranking score. We discuss how we did this in subsection 2.5.1.
Another difficulty was getting the universities for the Nobel Prize winners. More on this issue can be found in subsection 2.5.2.

\subsubsection{Matching of universities}
Not all the universities of the people (found on DBpedia) were an exact match with some university from the Shanghai rankings.
When taking a closer look at these, we discovered that some of them actually were in the list of Shanghai rankings, but with different spelling.
For this reason, we first applied some transformations on both the universities of the people and those from the Shangai rankings before matching them. These transformations include : setting everything to lowercase, removing everything in parentheses, removing all punctuation, removing the words 'university', 'college', 'state', 'of' and 'the' and removing all white spaces.
Universities which still not can be found in this way are just expected to have a ranking score of 0.

\subsubsection{Getting the universities for the Nobel Prize winners}

Another problem we encountered was while getting the universities for the Nobel Prize winners. Nobelprize.org does not contain the alma mater for the Nobel Prize winners. Instead, they store the unversity or research institution the Nobel Prize winner was affiliated with when winning the prize. For this reason we had to get the alma maters for the Nobel Prize winners from DBpedia. As Nobelprize.org does not have a \textit{sameAs} link with DBpedia, this also has to be done with simple name matching on the first and last name of the Nobel Prize winner.
However, DBpedia does not have an alma mater for 580 out of the total of 866 Nobel Prize winners. Since we already have to deal with a situation in which we have a lot of negative examples and a few positive examples, we want to use as many Nobel Prize winners as possible. For this reason, we settled for the universities stored on Nobelprize.org for all Nobel Prize winners for whom we could not find their alma mater on DBpedia.
However, there are still 191 Nobel Prize winners for whom we could not find a university both on DBpedia and Nobelprize.org and that we omit for this reason.

\section{Data analysis}
\label{sec:analysis}

\subsection{Quality assessment}
\label{ssec:quality}
\todo[inline]{Alinea over hoe we data samen zetten. Name matching, problemen hierbij, Aantal entries in training dataset en research set, aantal positieve voorbeelden t.o.v. negatieve voorbeelden (en vermelden dat dit een probleem is en hoe we dit oplossen/compenseren), andere statistieken i.v.m. data, outliers}

\subsection{Learning a model}


\todo[inline]{Statistieken i.v.m training en research sets hier?}

\todo[inline]{figuren hier ter visualisatie data}

\todo[inline]{Sectie over difficulties (training data veel negatieve en weinig positieve examples, links die er niet zijn}

\section{Results}
\label{sec:results}

\section{Conclusion}

\section{Future work}

We could broaden our training dataset by not only including people that won a Nobel Prize, but also gathering people who were ever nominated for one. They are not available on the LoD from Nobelprize.org but can be found on the website itself\cite{nominated}.
Our dataset could maybe also be enlarged with individuals who nominate other candidates. These people might be interesting because they are usually also influencial in the field of study from their proposed candidate.


\todo[inline]{Nog iets extra?}

% use section* for acknowledgment
\ifCLASSOPTIONcompsoc
  % The Computer Society usually uses the plural form
  \section*{Acknowledgments}
  We would like to thank our supervisors professor Bettina Berendt and Sebastijan Dumancic for their enthousiasm and support.
\else
  % regular IEEE prefers the singular form
  \section*{Acknowledgment}
\fi

% if have a single appendix:
%\appendix[Proof of the Zonklar Equations]
% or
%\appendix  % for no appendix heading
% do not use \section anymore after \appendix, only \section*
% is possibly needed

% use appendices with more than one appendix
% then use \section to start each appendix
% you must declare a \section before using any
% \subsection or using \label (\appendices by itself
% starts a section numbered zero.)
%

% Can use something like this to put references on a page
% by themselves when using endfloat and the captionsoff option.
\ifCLASSOPTIONcaptionsoff
  \newpage
\fi

% trigger a \newpage just before the given reference
% number - used to balance the columns on the last page
% adjust value as needed - may need to be readjusted if
% the document is modified later
%\IEEEtriggeratref{8}
% The "triggered" command can be changed if desired:
%\IEEEtriggercmd{\enlargethispage{-5in}}

% references section

% can use a bibliography generated by BibTeX as a .bbl file
% BibTeX documentation can be easily obtained at:
% http://www.ctan.org/tex-archive/biblio/bibtex/contrib/doc/
% The IEEEtran BibTeX style support page is at:
% http://www.michaelshell.org/tex/ieeetran/bibtex/
%\bibliographystyle{IEEEtran}
% argument is your BibTeX string definitions and bibliography database(s)
%\bibliography{IEEEabrv,../bib/paper}
%
% <OR> manually copy in the resultant .bbl file
% set second argument of \begin to the number of references
% (used to reserve space for the reference number labels box)
\printbibliography


\appendices
%\onecolumn

\clearpage
\section{Discussion on work productivity}
\label{sec:productivity}

Whether we can use different proxy measures for a certain feature is a sensitive issue. For this reason, we here provide a more thorough discussion of the matter. First, we discuss why we have to use different categories of people for building the model and training it. This introduces the problem of having to proxy a feature using  different measures. Next, we discuss why we believe the chosen measures might be a good proxy for the productivity feature. Finally, we give a weakened version of the research question that is now more appropriately answered.

\subsection{Different datasets}

The assignment was to find a research question revolving around the knowledge base, enriched with information from other sources. Because we found an interesting knowledge base on Nobel Prizes, the posed research question wasn't a far fetch. The problem is that in order to build a model on who would win a Nobel Prize, we need positive and negative examples. The positive examples are the Nobel laureates, which come from a pool of scientists. All other scientists naturally serve as negative examples. Since a politician isn't eligible to receive such prize, we can't use them as negative examples. It is thus inevitable to use different examples in the training set and the testing set.

\subsection{Work productivity}

Apart from university rankings and popularity measures, work productivity seems like a reasonable feature to use in order to assess eligibility to receive an award. We reason that hard work will more likely result in making some scientific breakthrough, which increases the chances of someone winning a Nobel Prize. The productivity feature is not trivial to define, however. Furthermore, even if we can define it, all information needed to compute it has to be available for as many training and testing instances as possible. We therefore view all examples as \emph{persons} rather than \emph{scientists} or \emph{politicians}, having certain properties. How we define the work productivity property is the only difference between both datasets. For scientists, the number of publications seems like a decent way of measuring how productive they've been. For politicians, the most resemblant feature would have been the number of legislations they managed to get accepted. This information isn't publicly available, however, so we looked at something else. Coming straight from the ToE knowledge base, the number of speeches at european parliament seemed a good proxy measure. Both are numeric properties which we can normalise to $[0, 100]$.

\subsubsection{Work productivity duality}
An interesting duality exists in both measures. Having a lot of publications over a mediocre amount can be recognised by the model as either a \emph{good} or \emph{bad} property. It can for example mean that someone is just trying to publish a lot of less significant results, while a mediocre amount of significant results might be more important. For speeches the same happens, where a lot of speeches might just as well mean that a politician is only talking, without actually \emph{doing} much. This re\"inforces our belief that both might be a good proxy for the same feature.

\subsection{Weakened research question}
A more appropriate version of the research question is then
\begin{center}
	\emph{\textbf{\textsc{IF}} the number of publications and the number of speeches in European Parliament are reliable proxies for the work productivity of scientists and politicians respectively \textbf{\textsc{THEN}} which European politicians have a high chance of receiving a Nobel Prize?}
\end{center}




\clearpage
\section{Retrieving data from Google Scholar}
\label{sec:googlescholar}

Retrieving data from Google Scholar is a laborous task.
This appendix provides our experience and also our tips on how to scrape Google Scholar.


\clearpage
\section{Linked open data}
\label{sec:lod}
\subsection{Sparql endpoints}
\begin{table}[H]
\centering
\begin{tabular}{l|l}
	\textbf{\textsc{Source}} & \textbf{\textsc{Sparql endpoint}} \\ \hline
	\rule{0pt}{4mm}DBPedia &  \\
	Year of birth & \textbf{DBPedia} \\
	University ranking & \textbf{DBPedia} + Shanghai Ranking 2015\\
	Work Productivity & Google Scholar, \textbf{ToE}\\
	Popularity & Facebook\\
	Nobel Prize& \textbf{Nobelprize.org}
\end{tabular}
\caption{Overview of used features and their respective sources.}
\end{table}
\subsection{Difficulites with LoD}
In this section we discuss our difficulties that we experienced with linked open data (LoD). These are worth mentioning because they influenced our results.

\subsubsection{The language barrier}
The great thing about LoD is that it comes in different languages, what means that you have a bigger and a broader information source. 

\subsubsection{LoD is sometimes broken}


% you can choose not to have a title for an appendix
% if you want by leaving the argument blank
%\input{appendices/queries}


%\section{Source code}
%Appendix two text goes here.
%\lstinputlisting[language=Java, caption=Opdracht 1]{../src/Main.java}
%\lstinputlisting[language=Java, caption=NobelPrice]{../src/NobelPrize.java}
%\lstinputlisting[language=Perl, caption=getRankings.pl]{../scripts/getRankings.pl}

\end{document}

