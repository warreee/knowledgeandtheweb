\documentclass[11pt,a4paper]{article}
\usepackage[latin1]{inputenc}
\usepackage[english]{babel}
\usepackage{amsmath}
\usepackage{amsfonts}
\usepackage{amssymb}
\usepackage{url}
\usepackage{hyperref}
\author{
  Katrien Laenen\\
  \and
  Gust Verbruggen\\
  \and
  Ward Schodts
}
\title{Knowledge \& the web: exercise session 4}
\begin{document}
\maketitle
\section{Interesting knowledge bases}
\begin{enumerate}

\item Musicbrainz
\item \url{data.nobelprize.org}
\item \url{http://datahub.io/dataset/zdb}
\item \url{www.linkedtv.eu}

\end{enumerate}

\section{Interesting questions}

All questions involve speakers from the ToE knowledge base. by combining it with other knowledge bases, we can infer other interesting properties about them. 


\subsection{How many and which European Parliament members were involved with criminal activities?}

DBpedia has information on criminal activities for people.

\subsection{Which politicians have good chances of receiving a Nobel Price?}

The nobelprize knowledgebase has information about nobel prize winners. By searching for similarities (trough for example DBpedia), we can maybe find out which politicians have a (high) chance of receiving a nobel prize.

\subsection{Which song came out on the birthday of a specific politician?}

ToE has birth dates for speakers, so we only have to combine it with artist information from musicbrainz to find our answer.

\subsection{Which politicians have a high chance of being murdered?}

Freebase contains the cause of death for deceased people, so yet again by searching for similarities with the European Parliament speakers, we can see who has a high chance of being murdered.

\subsection{Which European politician have published bibliographic material?}

The ZDB knowledge base contains information about bibliographic resources in many languages, chances are therefore high that at least one of the European Parliamentary speakers has published such material.

\subsection{Which politician have the most tv appearances?}

The LinkedTV project aims to annotate media fragments with metadata, such as \emph{faces} that appear in them. We'd like to find out if one of our speakers has made such television appearance.

\end{document}